\documentclass[12pt]{article}

\usepackage[fleqn]{amsmath}
\usepackage{amssymb}
\usepackage{amsthm}
\usepackage{graphicx}
\usepackage{float}
\theoremstyle{plain}     %------- 'regular' theorem types
\newtheorem{thm}{Theorem}[section]
\newtheorem{cor}[thm]{Corollary}
\newtheorem{lemma}[thm]{Lemma}
\newtheorem{prop}[thm]{Proposition}
\newtheorem{example}[thm]{Example}
\newtheorem{exer}[thm]{Exercise}
\newtheorem{define}[thm]{Definition}
\usepackage[top=2cm, left=2cm, right=2cm]{geometry} 
\usepackage{parskip}
\setlength{\parindent}{0in}
\usepackage{floatflt}
\usepackage{multicol}
\usepackage{tabu}
\usepackage[hidelinks]{hyperref}
\hypersetup{
	urlcolor=blue}

%%%%%%%%%%%%%%%%%%%%%%
%%%%%%%%%%%%%%%%%%%%%%%
%%%%%%%%%%%%%%%%%%%%%%%%
\begin{document}
\large
%subject
City Semester  \hspace{8cm} Name:\makebox[6cm]{\hrulefill}
\\
%specific topic
Measuring Diversity\\
\normalsize 
%\emph{Complete all work on a separate sheet of paper with exercises clearly labeled and all reasoning and work given.}\\[.5cm]
%\emph{Show all work for full credit.}\\
\section*{Three measures}
The 	Shannon Index (H) is calculated as $$H=\exp\left(-\sum\limits_{i=1} ^s p_i \ln(p_i)\right)$$.\\
The Simpson Index (D) is calculated as $$D=\frac{1}{\sum\limits_{i=1}^s p_i^2}$$
where $s$ is the number of different species and $p_i$ is the relative frequency of the $i$th species, or $\dfrac{n_i}{N}$ with $n_i$ being the number of the $i$th species and $N$ being the total number of organisms.
The previous two measures are often used to determine species diversity in a given community. We will examine how they differ from each other by looking at some data.\\[.5cm]
It is also sometimes useful to measure if two communities are similar to one another. 
Soresson's Coefficient (CC) is calculated as $$CC=\frac{2C}{S_1 + S_2}$$
where $C$ is the number of species in common and $S_1$ and $S_2$ are the number of species in each community.
\section*{examples}
{\renewcommand{\arraystretch}{1.5}%
\begin{tabular}{l|c|c|c|c|c|c}
Species & number of individuals ($n_i$) & rel. freq. ($n_i/N=p_i$) & $p_i^2$ & $p_i \ln p_i$\\
\hline
 Grasshopper green & 6 & $6/27=0.\overline{2}$ & 0.049 & 	-0.334\\
 \hline
 Grasshopper brown & 5 &&&\\
	\hline
 Large blue butterfly & 1 &&&\\
\hline 
 Small blue butterfly & 3 &&&\\
\hline 
 Red and blue beetle & 12 &&&\\

\end{tabular}
\begin{enumerate}
	\item Complete the table.
	\item Calculate $H$ and $D$.
	\item What are the values of $H$ and $D$ if there only one species is collected?
	\item What are the values of $H$ and $D$ if there are five examples of five species are collected for a total of 25 organisms.
	
\end{enumerate}
	\begin{tabular}{l|c|c|c|c|c|c}
Species & number of individuals ($n_i$) & rel. freq. ($n_i/N=p_i$) & $p_i^2$ & $p_i \ln p_i$\\
\hline
 Black Wasp &12 &  & & \\
 \hline
 Purple Wasp & 21 &&&\\
	\hline
 Bee & 5 &&&\\
\hline 
 Green Grasshopper & 25 &&&\\
\hline 
 Large Blue & 17 &&&\\

\end{tabular}
\begin{enumerate}
	\item Complete Table.
	\item Calculate $H$ and $D$.
	\item Which of the two communities is more diverse?
	\item How similar are the two communities?
\end{enumerate}
\end{document} 