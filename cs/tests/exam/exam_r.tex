\documentclass[12pt]{article}

\usepackage[fleqn]{amsmath}
\usepackage{amssymb}
\usepackage{amsthm}
\usepackage{graphicx}
\usepackage{float}
\theoremstyle{plain}     %------- 'regular' theorem types
\newtheorem{thm}{Theorem}[section]
\newtheorem{cor}[thm]{Corollary}
\newtheorem{lemma}[thm]{Lemma}
\newtheorem{prop}[thm]{Proposition}
\newtheorem{example}[thm]{Example}
\newtheorem{exer}[thm]{Exercise}
\newtheorem{define}[thm]{Definition}
\usepackage[top=2cm, left=2cm, right=2cm]{geometry} 
\usepackage{parskip}
\setlength{\parindent}{0in}
\usepackage{floatflt}
\usepackage{multicol}
\usepackage{tabu}
\usepackage[hidelinks]{hyperref}
\hypersetup{
	urlcolor=blue}

%%%%%%%%%%%%%%%%%%%%%%
%%%%%%%%%%%%%%%%%%%%%%%
%%%%%%%%%%%%%%%%%%%%%%%%
\begin{document}
\large
%subject
City Semester  \hspace{8cm} Name:\makebox[6cm]{\hrulefill}
\\
%specific topic
Semester Review\\
\normalsize 
%\emph{Complete all work on a separate sheet of paper with exercises clearly labeled and all reasoning and work given.}\\[.5cm]
\emph{Show all work for full credit.}
\section{Descriptive Statistics}
\begin{enumerate}
\item Given the data: 1,3,4,4,6,8,12,13,20. 
	\begin{enumerate}
		\item Give the five numbers summary and a sketch of a box plot that describes this data.
		\item Give the mean and the standard deviation of the data set.
		\item What number(s) give a description of the center of the data? How does a skew data set affect these measures?
		\item What number(s) give a description of the spread of the data? How does a skew data set affect these measures?
	\end{enumerate}
	\item Draw a symmetric distribution and a skew left distribution the same median. Where are the means of each distribution?
	\item What is the purpose of generating a scatterplot?
	\item Given a bivariate data set you generate a least squares regression line $\hat{y}=4x+3$.
		\begin{enumerate}
			\item What $y$ value is predicted for an $x$ value of 10?
			\item If one of the data points is $(4,20)$, what is the residual at $x=4$?
			\item What is the purpose of looking at the residual plot of a least squares regression line?
		\end{enumerate}
		\item Is it believable that there is a positive correlation between number hours spent per week playing video games and SAT math scores? Explain.
		\item When generating a least squares regression line, what does the $r$ value tell you? what about $r^2$?
		
\end{enumerate}
\section{Exponentials and Logs}
\begin{enumerate}
	\item When looking at data how can you distinguish linear data from exponential data?
	\item If a population of moles is growing at 7\% a year give a function expressing the population in terms of $t$ measured in years.
	\item Give a sketch of the following functions, clearly marking at least one point, and any asymptotes.
		\begin{enumerate}
			\item $f(x)=2^{-x}$
			\item $g(x)=5e^x$
			\item $h(x)=-\dfrac{1}{3}^x$
		\end{enumerate}
	\item Let $f(x)=2^x-4$.
		\begin{enumerate}
			\item What is the range of $f$?
			\item What is the domain of $f$?
			\item Give the equation of the asymptote?
		\end{enumerate}
	\item The half-life of a drug in blood stream is 10 hours. If you take a 100mg dose at noon, how much is left over in your bloodstream at midnight?
	\item Evaluate the following expression without a calculator.
		\begin{enumerate}
			\item $\log_2 32$
			\item $\log 1000$
			\item $\ln \sqrt{e^2}$
			\item $\log \dfrac{1}{100}$
		\end{enumerate}
	\item Give a sketch of both $f(x)=e^x$ and $g(x)=\ln x$ on the same axes. This illustrates that $f$ and $g$ have what geometric relation to one another?
	\item Given $f(x)=\log (x-2)$
		\begin{enumerate}
			\item What is the domain of $f$?
			\item What is the range of $f$?
			\item Give the equation of any asymptotes.
		\end{enumerate}
	\item If a population of 200 buffalo is growing by 5\% each year, when will the population reach 1200 buffalo?
\end{enumerate}
\section{Sequences and Series}
Review your test.
\newpage


\section{Combinatorics}
\begin{enumerate}
	\item Simplify the following expressions involving factorials without a calculator.
		\begin{multicols}{2}
		\begin{enumerate}
			\item $\dfrac{10!}{8!}$\\
			\item $2!3!$\\
			\item $(2\cdot 3)!$\\
			\item $3!+4!$\\
			\item $(3+4)!$\\
			\item $\dfrac{n!}{(n+1)!}$\\
			\item $\dfrac{n!}{2(n-1)!}$\\
		\end{enumerate}
		\end{multicols}
		\item What errors do parts b-e in the previous problem warn against?
		\item You select a password consisting of 4 different letters or 4 different digits, how many different passwords are possible?\\
		\item You have 7 books. You want to put 5 of them on the shelf. How many different arrangements can be made?
		\item 8 people meet at a party. Each pair of people shakes hands, how many handshakes were there?
		\item You have 3 red, 3 blue and 2 yellow flags. You can send messages by placing the 8 flags in sequence. Each different sequence is a different message. How many different messages can you send?
\end{enumerate}
\end{document} 