\documentclass[12pt]{article}

\usepackage[fleqn]{amsmath}
\usepackage{amssymb}
\usepackage{amsthm}
\usepackage{graphicx}
\usepackage{float}
\theoremstyle{plain}     %------- 'regular' theorem types
\newtheorem{thm}{Theorem}[section]
\newtheorem{cor}[thm]{Corollary}
\newtheorem{lemma}[thm]{Lemma}
\newtheorem{prop}[thm]{Proposition}
\newtheorem{example}[thm]{Example}
\newtheorem{exer}[thm]{Exercise}
\newtheorem{define}[thm]{Definition}
\usepackage[top=2cm, left=2cm, right=2cm]{geometry} 
\usepackage{parskip}
\setlength{\parindent}{0in}
\usepackage{floatflt}
\usepackage{multicol}
\usepackage{tabu}
\usepackage[hidelinks]{hyperref}
\hypersetup{
	urlcolor=blue}

%%%%%%%%%%%%%%%%%%%%%%
%%%%%%%%%%%%%%%%%%%%%%%
%%%%%%%%%%%%%%%%%%%%%%%%
\begin{document}
\large
%subject
Geometry  \hspace{8cm} Name:\makebox[6cm]{\hrulefill}
\\
%specific topic
Quiz\\
\normalsize 
%\emph{Complete all work on a separate sheet of paper with exercises clearly labeled and all reasoning and work given.}\\[.5cm]
\emph{Show all work for full credit.}
\begin{enumerate}
	\item Given a right triangle with leg lengths 15 and 17, find the length of the hypotenuse and the measure of its interior angles.\\[3cm]
	\item Given a right triangle with hypotenuse 15 and an acute angle of $15^\circ$, find the lengths of the two legs.\\[3cm]
	\item What acute angle does $y=\dfrac{2}{3}x+1$ make with the line $y=5$?
	\end{enumerate}
	

\end{enumerate}
	
\end{document} 