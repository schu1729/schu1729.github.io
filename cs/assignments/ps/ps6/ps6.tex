\documentclass[12pt]{article}

\usepackage[fleqn]{amsmath}
\usepackage{amssymb}
\usepackage{amsthm}
\usepackage{graphicx}
\usepackage{float}
\theoremstyle{plain}     %------- 'regular' theorem types
\newtheorem{thm}{Theorem}[section]
\newtheorem{cor}[thm]{Corollary}
\newtheorem{lemma}[thm]{Lemma}
\newtheorem{prop}[thm]{Proposition}
\newtheorem{example}[thm]{Example}
\newtheorem{exer}[thm]{Exercise}
\newtheorem{define}[thm]{Definition}
\usepackage[top=2cm, left=2cm, right=2cm]{geometry} 
\usepackage{parskip}
\setlength{\parindent}{0in}
\usepackage{floatflt}
\usepackage{multicol}
\usepackage{tabu}
\usepackage[hidelinks]{hyperref}
\hypersetup{
	urlcolor=blue}

%%%%%%%%%%%%%%%%%%%%%%
%%%%%%%%%%%%%%%%%%%%%%%
%%%%%%%%%%%%%%%%%%%%%%%%
\begin{document}
\large
%subject
City Semester  \hspace{8cm} Name:\makebox[6cm]{\hrulefill}
\\
%specific topic
Problem Set \#6\\
\normalsize 
%\emph{Complete all work on a separate sheet of paper with exercises clearly labeled and all reasoning and work given.}\\[.5cm]
\emph{Show all work for full credit.}\\
\begin{enumerate}
	\item Generate the first 5 terms of each of the following sequences.
		\begin{enumerate}
			\item $a_n= (-1)^n 2n^2$\\[2cm]
			\item $b_n= 3n-1$\\[2cm]
			\item $c_n= 3\left(-\dfrac{1}{2}\right)^n$\\[2cm]
			\item $d_n= d_{n-1}^2-1$ and $d_1=2$\\[2cm]
		\end{enumerate}
\item Given an arithmetic sequence that increases by 4 each term, and starts at -5, find the 100th term of this sequence.\\[3cm]
\item Given a geometric sequence with ratio 1/3 and first term 81, find the 10th term of this sequence.\\[3cm]
\item Evaluate the following series.
\begin{multicols}{2}
	\begin{enumerate}
		\item $\sum\limits_{n=1}^{50} (3n+1)$\\[5.5cm]
		\item $\sum\limits_{n=1}^{20}2\left(\dfrac{3}{2}\right)^{n-1}$\\[5cm]
		\item $\sum\limits_{n=1}^{20}3\left(\dfrac{3}{4}\right)^{n-1}$\\[5cm]
		\item $\sum\limits_{n=1}^\infty 2\left(\dfrac{1}{3}\right)^{n-1}$\\[5cm]
		\item $\sum\limits_{n=1}^\infty 2\left(-\dfrac{1}{5}\right)^{n-1}$\\[5cm]
		\item $\sum\limits_{n=0}^\infty \left(\dfrac{4}{5}\right)^n$\\[5cm]
		
	\end{enumerate}
\end{multicols}
\newpage
\item Use the properties of logs to rewrite each of these expression in terms of $x$ and $y$ where $x=\log(A)$ and $y=\log(B)$.
	\begin{enumerate}
		\item $\log(A/B)$\\[1cm]
		\item $\log(\sqrt{A}/B)$\\[1cm]
		\item $\log(A^3B^5)$\\[1cm]
		\item $\log(2\sqrt{A^3}/B^5)$\\[1cm]
		
	\end{enumerate}
\item The doubling time of a population of gnats is 15 hours. Assuming the population can be modeled exponentially give the model of the populations of gnats if you know at time 7 hours there were 120 gnats.\\[4cm]

\item The half life of caffeine in the bloodstream is 4 hours. Assume the caffeine decays exponentially. If I drink a cup of coffee every 6 hours for the next 2 weeks approximately how much caffeine is in my bloodstream during that second week?
	
\end{enumerate}
	
\end{document} 