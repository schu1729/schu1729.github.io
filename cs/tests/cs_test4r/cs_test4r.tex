\documentclass[12pt]{article}

\usepackage[fleqn]{amsmath}
\usepackage{amssymb}
\usepackage{amsthm}
\usepackage{graphicx}
\usepackage{float}
\theoremstyle{plain}     %------- 'regular' theorem types
\newtheorem{thm}{Theorem}[section]
\newtheorem{cor}[thm]{Corollary}
\newtheorem{lemma}[thm]{Lemma}
\newtheorem{prop}[thm]{Proposition}
\newtheorem{example}[thm]{Example}
\newtheorem{exer}[thm]{Exercise}
\newtheorem{define}[thm]{Definition}
\usepackage[top=2cm, left=2cm, right=2cm]{geometry} 
\usepackage{parskip}
\setlength{\parindent}{0in}
\usepackage{floatflt}
\usepackage{multicol}
\usepackage{tabu}
\usepackage[hidelinks]{hyperref}
\hypersetup{
	urlcolor=blue}

%%%%%%%%%%%%%%%%%%%%%%
%%%%%%%%%%%%%%%%%%%%%%%
%%%%%%%%%%%%%%%%%%%%%%%%
\begin{document}
\large
%subject
City Semester  \hspace{8cm} Name:\makebox[6cm]{\hrulefill}
\\
%specific topic
Review of Sequences and Series\\
\normalsize 
%\emph{Complete all work on a separate sheet of paper with exercises clearly labeled and all reasoning and work given.}\\[.5cm]
\emph{Show all work for full credit.}
\begin{enumerate}
	\item Find $a_5$ for the following sequences.
		\begin{enumerate}
			\item $a_n=2n-4$
			\item $a_n=(-1)^n \dfrac{3}{2^n}$
			\item $a_n=n(n-1)(n-2)(n-3)/(n+1)$
			\item $a_3=2$ and $a_n=a_{n-1}^2+na_{n-1}$
		\end{enumerate}
	\item The sequence is $b_n$ is arithmetic. $b_3=12$ and $b_8=-3$. Find the general term $b_n$ and then find the 111$^{\text{th}}$ term.	
	\item The sequence $c_n$ is geometric. $c_2=3$ and $c_5=\dfrac{3}{8}$. Find the general term $c_n$, and find tenth term of the sequence.
	\item Find the 50th term of the arithmetic sequence whose common difference 12 and whose second term is 10.
	\item Find the ninth term of the geometric sequence whose common ratio is 1/3 and whose 3rd term is 8.
	\item Write the following series using sigma notation.
		\begin{enumerate}
			\item $2+3+4+5+\ldots +98+99$
			\item $-3 + 6 -12+24-48+\ldots -768$
			\item $5+1-3-7-11-15-\ldots -195$
		\end{enumerate}
	\item Do some exercises summing arithmetic and geometric series from the textbook.
	\item You deposit 1000 dollars a year for 10 years in a bank account paying 6\% compounded semi-annually. How much will your account have after 10 years?
	\item You receive 5000 dollars a year for 10 years. If you discount these payments at 3\%, what is the present value of these cash flows?
	\item If you will receive 10 payments of 5000 dollars each year starting 3 years from now, what is the present value of these cash flows using a discount factor of 4\%?
	\item What is the payment on a 12 year mortgage with annual payments, a principal amount of \$100,000 and an interest rate of 4\%?
	\item What is the payment on a 15 year mortgage with monthly payments, a principal amount of \$250,000 and an interest rate of 3.75\%?
\end{enumerate}
	
\end{document} 