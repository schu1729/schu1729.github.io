\documentclass{amsbook}
\usepackage{graphicx} % Required for inserting images
\usepackage{float}
\usepackage[colorlinks=true,linkcolor=blue]{hyperref}
\usepackage{amssymb}
\usepackage{enumerate}
\usepackage{mathtools}
\usepackage{tikz}
\usepackage[answerdelayed]{exercise}
\usepackage{floatflt}
\usepackage{multicol}
\usepackage{subcaption}
\usepackage{tikz}
\usepackage{tkz-euclide}
\usetikzlibrary{trees}
\usepackage[backend=biber,style=alphabetic,sorting=ynt]{biblatex}
\addbibresource{prealg.bib}

\setlength{\parindent}{0pt}
\setlength{\listparindent}{0pt}
\renewcommand{\ExerciseHeaderTitle}{\quad\ExerciseTitle}
\renewcommand{\AnswerHeader}{\medskip\centerline{\emph{\ExerciseName\ \ExerciseHeaderNB} \smallskip}}
\renewcommand{\ExerciseHeader}{\medskip\ExerciseHeaderDifficulty{\emph{\ExerciseName\ \ExerciseHeaderNB\ExerciseHeaderTitle} \newline}}
\renewcommand{\QuestionNB}{(\alph{Question})\ }
\renewcommand{\Re}{\operatorname{Re}}
\renewcommand{\Im}{\operatorname{Im}}

\begin{document}

\title{Mathematics:\\
    \Large After Arithmetic}


\author{S. Chu}

\maketitle
\tableofcontents
\chapter*{Preface}
The idea for the following pages is to capture some of the activities we tried out during our Riverdale Math Circle sessions. The intended audience is those students who have learned arithmetic and are ready to develop their problem solving skills and thinking skills in the lead up to the study of algebra.

Problems are roughly categorized by difficulty level and notated via *, **, or *** in increasing difficulty. We have adopted problems from memory, and from years of working with the material. The following references were consulted for problems as well as other considerations such as narrative structure in introducing various topics and thematic development of the material. \cite{mcirc}, \cite{aops}, \cite{hpn}, [more later].
\chapter{Counting Stones}
\section*{Counting}

\section*{Factors}

\section*{Primes}

\section*{Counting Rearrangements}


\chapter{Length \& the Number Line}


\chapter{Where Are We?}

\chapter{Getting from A to B}

\chapter{Letting Things Vary}

%%%%%
%%%%%
%End of problems of the book.
%%%%%%%
%%%%%

\chapter*{Answers to Selected Exercises}
\shipoutAnswer

\printbibliography
\end{document}
