\documentclass[12pt]{article}

\usepackage[fleqn]{amsmath}
\usepackage{amssymb}
\usepackage{amsthm}
\usepackage{graphicx}
\usepackage{float}
\theoremstyle{plain}     %------- 'regular' theorem types
\newtheorem{thm}{Theorem}[section]
\newtheorem{cor}[thm]{Corollary}
\newtheorem{lemma}[thm]{Lemma}
\newtheorem{prop}[thm]{Proposition}
\newtheorem{example}[thm]{Example}
\newtheorem{exer}[thm]{Exercise}
\newtheorem{define}[thm]{Definition}
\usepackage[top=2cm, left=2cm, right=2cm]{geometry} 
\usepackage{parskip}
\setlength{\parindent}{0in}
\usepackage{floatflt}
\usepackage{multicol}
\usepackage{tabu}
\usepackage[hidelinks]{hyperref}
\hypersetup{
	urlcolor=blue}

%%%%%%%%%%%%%%%%%%%%%%
%%%%%%%%%%%%%%%%%%%%%%%
%%%%%%%%%%%%%%%%%%%%%%%%
\begin{document}
\large
%subject
City Semester  \hspace{8cm} Name:\makebox[6cm]{\hrulefill}
\\
%specific topic
Problem Set \#5\\
\normalsize 
%\emph{Complete all work on a separate sheet of paper with exercises clearly labeled and all reasoning and work given.}\\[.5cm]
\emph{Show all work for full credit.}\\
\begin{enumerate}
	\item Recall that to say that $y=\log_b x$ also implies that $b^y=x$. Evaluate the following expressions without a calculator.
	\begin{multicols}{2}
		\begin{enumerate}
			\item $\log_2 64$
			\item $\log_2 \frac{1}{32}$
			\item $\log_2 \sqrt{8}$
			\item $\log_2 \frac{\sqrt{32}}{\sqrt[3]{2}}$
			\item $\ln e^2$
			\item $\ln \sqrt[3]{e^4}$
			\item $2\log 10 + 3\ln e^3$
		\end{enumerate}
	\end{multicols}
	\item Remember that the exponential $y=b^x$ is the inverse of the logarithm base $b$, in other words $\displaystyle b^{\log_b u}=u$ and $\log_b b^u=u$. Evaluate the following without a calculator.
	\begin{multicols}{2}		
		\begin{enumerate}
			\item $\log 10^{2x}$
			\item $10^{\log 2x}$
			\item $\log 10^{x^2+2x+1}$
			\item $e^{3\ln 5x}$
		\end{enumerate}	
	\end{multicols}	
	\item Use the idea of inverse functions to solve for $x$ in the following equations.
	\begin{multicols}{2}
		\begin{enumerate}
			\item $120 = 3(10)^x$
			\item $120 = 3(10)^{x+3}$
			\item $100 = \log (10x)$
			\item $100 = \log (25x^2)$
			\item $100 = 3e^{x+5}$
			\item $100 = \ln (25x^3)$
		\end{enumerate}			
	\end{multicols}
	\item Let $A(t)=1000(1.03)^t$.
		\begin{enumerate}
			\item If the function $A$ were supposed to model the amount of money $A$ in terms of time $t$ in years, give an explanation of the meaning of the numbers 1000 and 1.03 in the model.
			\item How much money is there at time 0?
			\item How long would it take to double the amount of money?
			\item How long would it take to triple the amount of money?
			\item How long would it take to have five times the original amount of money?
			\item How long would it take to have $n$ times the original amount of money?
		\end{enumerate}
\newpage

	\item Let $f(x)= \log_2 (x-5)$ and $g(x)=\log (x+2)+5$
		\begin{enumerate}
			\item Give the domain of $f$.
			\item Give the domain of $g$.
			\item Give the range of $f$.
			\item Give the range of $g$.
			\item On the same axes give a sketch of both $f$ and $g$ including their vertical asymptotes.
		\end{enumerate}
	\item Explain why the magnitude of earthquakes is measured on a logarithmic scale.
	\item What other phenomena are measured on a logarithmic scale?
\end{enumerate}
	
\end{document} 