\documentclass[12pt]{article}

\usepackage[fleqn]{amsmath}
\usepackage{amssymb}
\usepackage{amsthm}
\usepackage{graphicx}
\usepackage{float}
\theoremstyle{plain}     %------- 'regular' theorem types
\newtheorem{thm}{Theorem}[section]
\newtheorem{cor}[thm]{Corollary}
\newtheorem{lemma}[thm]{Lemma}
\newtheorem{prop}[thm]{Proposition}
\newtheorem{example}[thm]{Example}
\newtheorem{exer}[thm]{Exercise}
\newtheorem{define}[thm]{Definition}
\usepackage[top=2cm, left=2cm, right=2cm]{geometry} 
\usepackage{parskip}
\setlength{\parindent}{0in}
\usepackage{floatflt}
\usepackage{multicol}
\usepackage{tabu}
\usepackage[hidelinks]{hyperref}
\hypersetup{
	urlcolor=blue}
%%%%%%%%%%%%%%%%%%%%%%
%%%%%%%%%%%%%%%%%%%%%%%
%%%%%%%%%%%%%%%%%%%%%%%%
\begin{document}
\large
%subject
City Semester  \hspace{8cm} Name:\makebox[6cm]{\hrulefill}
\\
%specific topic
Harper's Index Assignment\\
\normalsize 
%\emph{Complete all work on a separate sheet of paper with exercises clearly labeled and all reasoning and work given.}\\[.5cm]
%\emph{Show all work for full credit.}\\
\begin{enumerate}
	\item Identify a set of 3 statistics that you find interesting from each index. Explain what you find interesting about each one.
	\item Mathematics often has a claim to being the one eternally true subject that one can study. This idea is often abused to support a particular narrative with `irrefutable' facts. Find an example of a particular narrative that is being told by each index. Explain how the author is trying to support a particular point of view.
	\item Create your own index about nyc of at least 10 statistics. Feel free to use any sources, but make sure to note them down. The following are good places to find some stats: \href{http://www.nycedc.com/resources/economic-research-analysis/economic-snapshots}{a.Economic stats}, \href{https://data.cityofnewyork.us/}{b. Lots of data sets here but requires some work to get stats from it}.
\end{enumerate}
	
\end{document} 