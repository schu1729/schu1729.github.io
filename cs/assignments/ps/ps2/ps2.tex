\documentclass[12pt]{article}

\usepackage[fleqn]{amsmath}
\usepackage{amssymb}
\usepackage{amsthm}
\usepackage{graphicx}
\usepackage{float}
\theoremstyle{plain}     %------- 'regular' theorem types
\newtheorem{thm}{Theorem}[section]
\newtheorem{cor}[thm]{Corollary}
\newtheorem{lemma}[thm]{Lemma}
\newtheorem{prop}[thm]{Proposition}
\newtheorem{example}[thm]{Example}
\newtheorem{exer}[thm]{Exercise}
\newtheorem{define}[thm]{Definition}
\usepackage[top=2cm, left=2cm, right=2cm]{geometry} 
\usepackage{parskip}
\setlength{\parindent}{0in}
\usepackage{floatflt}
\usepackage{multicol}
\usepackage{tabu}
\usepackage[hidelinks]{hyperref}
\hypersetup{
	urlcolor=blue}

%%%%%%%%%%%%%%%%%%%%%%
%%%%%%%%%%%%%%%%%%%%%%%
%%%%%%%%%%%%%%%%%%%%%%%%
\begin{document}
\large
%subject
City Semester %\hspace{8cm} Name:\makebox[6cm]{\hrulefill}
\\
%specific topic
Problem Set \#1\\
\normalsize 
\emph{Complete all work on a separate sheet of paper with exercises clearly labeled and all reasoning and work given.}\\
%\emph{Show all work for full credit.}\\
\begin{enumerate}
	\item \includegraphics[scale=.1]{fathom.png} A company with 278 employees was concerned that there was a problem with their employees showing up late to work so management recorded how many minutes each person was late.  If they showed up early or on time no information was collected for that employee.  Open the Fathom file called `Late to Work' to see the data.
	\begin{enumerate}
		\item Create a dotplot (you do not need to copy 		the dot plot to your paper).
		\item Six people were late the same number of 			minutes. How many minutes late were these six 			people?
		\item Let’s say that the next day the lateness 			data is the same with one exception, the person 		who was the latest (180 minutes late) arrived 			at work on time.  What would the graph look 				under these circumstances? Click on that dot 			and delete it to see.
		\item What does part c tell you about the 				appropriateness of a dotplot for certain datasets?  
	\end{enumerate}
	
	\item \includegraphics[scale=.1]{fathom.png} Two schools reported the Math SAT scores of their seniors.  Open the Fathom file called “SAT Scores” to see the data.
	\begin{enumerate}
		\item How many seniors were in each school?
		\item Create a dotplot with the “Score” variable on the horizontal axis.  Then drag the name of the “School” variable to the vertical axis to split the graph by school (you do not need to copy the dot plot to your paper).  What was the range of SAT scores for each school?
		\item Describe the shape of the distribution for each school.
		\item Based just on what you see, which school’s seniors did better on the SAT?  Support your answer.
		\item You explained why one school did better on the SATs in part c.  Now suppose you were the headmaster of the other school.  Write a few sentences trying to convince someone that your SAT scores are better than the school you chose in c.
\end{enumerate}	 

	\item \includegraphics[scale=.1]{fathom.png} Go to the NYC open data portal \href{https://data.cityofnewyork.us/}{here}. Then:
		\begin{enumerate}
			\item Find a data set which will give you a distribution to examine.\\
			\item Download and format the data in excel so that you can paste the data into fathom.
			\item Generate a box plot and a histogram. Comment on the shape of the distribution, its center, and its spread. Lastly save the fathom file as ps1q3.
		\end{enumerate}
		 
		
		

\newpage

	\item Make up a data set of five data values in which the mode is negative and the mean is positive.


	\item Make up a data set of ten data values in which the mean is negative and the median is positive.


	\item \includegraphics[scale=.1]{fathom.png} The salaries for the 2013 Mets and the total payrolls for all MLB teams is in the file “MLB Salaries”

Note: when Fathom deals with really large or really small numbers, at some point it starts to report them in what’s called “engineering notation”, which is basically the same as scientific notation.  If you see 1.05e+08, that means $1.05 x 10^8$, or 105,000,000.  2.3e-06 would mean $2.3 x 10^{-6}$, or 0.0000023

	\begin{enumerate}
		\item Create a dotplot of salaries for the Mets.  What is the shape of the distribution
		\item Before calculating them, which do you think will be larger, the mean salary or the median salary and why?
		\item What is the mean salary of the Mets?  
		\item What is the median salary of the Mets?   
		\item If you read a newspaper story about the Mets the mean as a measure of center, what type of “spin” would the writer be trying to use?
		\item If you read a newspaper story about the Mets that used the median as a measure of center, what type of “spin” would the writer be trying to use?
		\item Create a separate dotplot of all teams’ payrolls.  What is the shape of the distribution?  
		\item Based on looking at the distribution of all teams, do you expect the mean and median to be closer together or further apart then they were for the Mets?  Then use calculations to find out if you were correct or not.
\end{enumerate}	 
	\item \hspace{-2em}\llap{$*$}\hspace{2em}Convert the following cartesian coordinates to polar coordinates:
		\begin{enumerate}
			\item $(-6,8)$\\
			\item $(3,-3\sqrt{3})$\\
			\item $(-4\sqrt{2},-4\sqrt{2})$\\
			\item $(3,1)$\\
		\end{enumerate}
	
	\item \hspace{-2em}\llap{$*$}\hspace{2em}Convert the following polar coordinates to cartesian coordinates.
		\begin{enumerate}
			\item $(3,3\pi/4)$\\
			\item $(10,11\pi/6)$\\
			\item $(2,\pi/2)$\\
			\item $(4,2)$\\
		\end{enumerate}
	
	\item \hspace{-2em}\llap{$*$}\hspace{2em}Simplify the following expressions involving $i=\sqrt{-1}$. 
		\begin{enumerate}
			\item $i^3$\\
			\item $i^{253}$\\
			\item $(1+i)(3-2i)$\\
			\item $(4-5i)(4+5i)$\\
		\end{enumerate}
\end{enumerate}
	
\end{document} 